% Reduce default margins
\setlength{\topmargin}{-.5in}
\setlength{\textheight}{9in}
\setlength{\oddsidemargin}{.125in}
\setlength{\textwidth}{6.25in}
\captionsetup[table]{justification=justified,singlelinecheck=false,labelformat=empty,skip=0pt}
\colorlet{shadecolor}{blue!20}
\graphicspath{ {./images/} }
\captionsetup[figure]{justification=justified,singlelinecheck=false,labelformat=empty,skip=0pt}

\chapter{Detailanalüüs}
Selles peatükis kirjeldatakse detailselt ja mittetehniliselt funktsionaalse allsüsteemi/registri paari, mille eskiismudelid esitati strateegilise analüüsi dokumendis.

\section{Autode funktsionaalse allsüsteemi detailanalüüs}
Järgnevalt kirjeldatakse detailselt ja mittetehniliselt autode funktsionaalse allsüsteemi toimimist.

\subsection{Kasutusjuhtude mudel}
Autode funktsionaalse allsüsteemi kasutusjuhtude diagramm (vt joonis 2).  \vspace{\pVert} \\ \hfill
\textcolor{red}{Punasega} viidatakse andmebaasioperatsioonidele, mis seisnevad ainult andmete lugemises. \textcolor{blue}{Sinisega} viidatakse andmebaasioperatsioonidele, mis tegelevad andmebaasis andmete muutmisega.

\begin{shaded}
	\subsubsection{\colorbox{lightgray}{\underline{Kasutusjuht: Tuvasta kasutaja}}}
	\textbf{Primaarne tegutseja:} Autode haldur, Juhataja, Klienditeenindaja, Klient – (edaspidi Subjekt). \\
	\textbf{Osapooled ja nende huvid:} Autode haldur, Juhataja, Klienditeenindaja, Klient: Soovivad siseneda süsteemi ja teha tegevusi neile antud volituste piires. \\
\end{shaded}